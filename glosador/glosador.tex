\documentclass[12pt]{beamer}
\usetheme{Metropolis}
\usepackage[utf8]{inputenc}
\usepackage[spanish]{babel}
\usepackage{amsmath}
\usepackage{amsfonts}
\usepackage{amssymb}
\usepackage{graphicx}
\author{Diego Alberto Barriga Martínez}
\title{Glosado automático}
%\setbeamercovered{transparent} 
\setbeamertemplate{navigation symbols}{} 
%\logo{} 
%\institute{} 
%\date{} 
%\subject{} 
\begin{document}

\begin{frame}
\titlepage
\end{frame}

\begin{frame}
\tableofcontents
\end{frame}


\begin{frame}{Etiquetadores automáticas}
\protect\hypertarget{etiquetadores-automuxe1ticas}{}

something

\begin{itemize}

\item
  Los etiquetadores automáticos con una tarea común del Procesamiento de
  lenguaje Natural (PLN)
\item
  Es usual que se utilizen métodos de \emph{Machine Learning (ML)} para
  su construcción

  \begin{itemize}

  \item
    En particular métodos basados en gráficas, por ejemplo Modelos
    Ocultos de Markov (\emph{Hidden Markov Models, HMM})
  \end{itemize}
\end{itemize}

\end{frame}

\begin{frame}{El lenguaje}
\protect\hypertarget{el-lenguaje}{}

\end{frame}

\begin{frame}{•}

\end{frame}


\end{document}
